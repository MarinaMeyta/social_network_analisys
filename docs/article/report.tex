\input{config}

\begin{document} 

 
\section*{АНАЛИЗ ТОНАЛЬНОСТИ ТЕКСТА С ПОМОЩЬЮ КАКИХ-ТО ТАМ КЛАССИФИКАТОРОВ}

В этом посте я покажу, как воспользоваться API анализа тональности в социальных медиа на русском языке. Одной из отличительных фич системы является возможность определять тональность по отношению к заданному объекту мониторинга. Проиллюстрирую на примере:

Мне нравится телефон X, но телефон Y ужасен\cite{cfl}.

Логично ожидать, что если мы интересуемся телефонами X, то хотим получить позитивную метку тональности. И негативную для телефона Y. Именно такой результат вы получите, введя это предложение на сайте www.mashape.com/dmitrykey/russiansentimentanalyzer (понадобится регистрация) и указав в качестве объектов X либо Y.

Ниже вы найдёте пример кода на Java, который позволяет легко подключиться к API, послать запрос и получить результат. Этот и другие примеры взаимодействия с нашими системами можно найти на нашем гитхабе.

% \renewcommand{\refname}{ЛИТЕРАТУРА}
\section*{ЛИТЕРАТУРА}
\bibliography{lit}

\end{document}
